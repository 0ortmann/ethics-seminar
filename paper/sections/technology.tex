\section{Facilitating Anonymous Trading}
\label{tech}

Many different ethical viewpoints have been taken and moral implications and consequences of consumer and seller actions have been discussed in the context of deep web shops.

The following section covers thoughts about offering technology and enabling others to trade anonymously on deep web platforms. Knowing what those other entities could do, is it morally justifiable to develop software or host platforms for anonymous trading? Different viewpoints are discussed. Two main perspectives are differentiated. Software developers and their employing companies on the one hand and anonymous market platform operators on the other hand. Business ethics are discussed for both sides.

\subsection{Running a Platform}

All traders on Silk Road were anonymous. The platform operators did not know with certainty, what legal obligations may apply to whom. The platform \emph{could} be used to for illegal business in some countries, but the operators cannot know for sure. They know absolutely nothing about the traders. All they know is, that is brings in money (the seller's fees). For Silk Road that was approximately 92k USD per month \cite{silkroad2013}. According to the shareholder theory \cite{shareholder}, a business has to obey the law. But as long true anonymity exists on the platform, it remains unclear what laws to apply. From a shareholders perspective, there is nothing wrong with running a platform like Silk Road.

Silk Road was more then aware of the commodities sold on the platform. It was clearly supporting drug business. Silk Road even offered categorization and search, tailored for drugs \cite{silkroad2013}. It is out of question, that everybody using or operating that platform did know what was going on. This knowledge makes it impossible to act along all stakeholders interests.

Governments and law enforcement agencies throughout the world are stakeholders to any kind of drug trading. And thus are very much interested in anonymous market places. The Silk Road bust by the FBI in 2013 gives clear evidence, that governments are indeed stakeholders to such platforms. Silk Road was a massive enabler for thousands to circumvent law enforcement while upholding illegal business. When acting along the stakeholder theory, it is clearly unethical to run a deep web business.

\subsection{Software Development}

Many different software tools are combined to enable anonymous trading on platforms like Silk Road. The Tor network is used to disguise connection origins. Cryptocurrencies are used for payments, because they can be exchanged without identity verification. The trading platform itself can be considered a software tool. Another essential tool would be a browser to access the platform. And there are definitely many more building blocks of software being used to make platforms like Silk Road happen.

From all the above mentioned software, only the trading platform is tailored for the deep web. All the other technologies are rather just used. They could as well be used for something completely unrelated to anonymous trading or the deep web. Is it ethical to develop software that could be used for deep web shops?

Unlike running a deep web platform, almost all tools used to stay anonymous have more general purposes. One can argue with the intended use case of some software. A similar argument as for cars may apply: a car can surely be used to run over and hurt somebody. But in general that is neither the \emph{only} nor the \emph{intended} use case for cars. Cryptocurrencies, the Tor network, or even internet browsers like Chrome have a multitude of legitimate use cases. Some software is not unethical, just because it could be abused for illegitimate actions. Or is it? That is a more general discussion, which I will not cover here. But it can at least be seen as questionable, to provide others with the tools to eventually do evil.