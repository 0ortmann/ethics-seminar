\section{Anonymous Online Trading}

%- different kinds of anon. who can be anon, for whom is that interesting? for whom is that bad? enables what? crime, no state surveillence...

Trading goods involves at least two entities. Think of them as vendor and consumer. They could again consist of more than one entity, but for simplicity reasons, vendor and consumer are each treated as one entity. More entities are involved, if the traders use trading platforms or payment providers. In the case of online trading, not only the direct participants have to be considered, but also for example the internet service providers (ISPs) or server hosters. Governments are also involved to some extent, as both -- vendor and consumer -- could live in countries where some regulations on trading are set up.
This section will elaborate on the question, what it means for a consumer or a vendor to be anonymous. The consequences for other holders of interest are discussed.

\subsection{Anonymous Entities}

% Dfferent Levels of anonimity, different entities that can be anonymous.
% \begin{itemize}
    % \item Anonymous Vendors
    % \item Anonymous Consumers
    % \item Anonymous Payments/Cashflow
    % \item Anonymous Platforms -> owner knows that shit goes on. Likes it? Runs platform anyways.
% \end{itemize}
% (nicht zu tief, was egtl anonym ist.)

\subsection{Trusting the Anonymity}

% There exist many different levels of anonymity. Many "weak spots" in the communication infrastructure pipe. 

% Es ist schwer anonymität zu schaffen / sogar zu gewährleisten.
% Meine def von anonymität: es ist mir "considerable effort" nicht möglich die identität der teilnehmer festzustellen. oder man nutzt halt einen öffentlichen inet zugang und gefälschte adressen um die ware zu erhalten.
% schwierigkeiten beim good exchange in beiden welten:
%leak von anonymität bei 
    %- geldtransfer
    %- warentransfer

% (Anon ist nun also irgendwie möglich. darauf aufbauend:)

\subsection{Effects of Anonymity during Trade}
% Warum könnte das schlecht sein? Aus wessen sicht?

%Anonymität als driver für Kriminalität, weil strafverfolgung nicht zu befürchten ist.
%-> ganz andere handels ziele
% Regulierung ist auch schutz. der fällt nun auch weg. könnte das trotzdem reguliert sein?

%Warum könnte das gut sein, aus wessen sicht?
% Anonymität als driver für vielfältigkeit. Diversity is nice to have. anonymity enables diversity. accountability disables it! paper cite
% sicherheit der trader vor einander und vor driteen: man braucht keine angst haben, (körperlich) abgezogen zu werden bei irgendwelchen shady deals. sicherheit vor strafverfolgung.
% vorteile der anonymität an sich: zb peinliche waren (nicht unbedingt illegale waren!), erotik

% allgemein, beide sichten betreffend: ganz andere waren werden ausgetauscht - gov findet das doof, alle anderen voll gut. silkroad examples
