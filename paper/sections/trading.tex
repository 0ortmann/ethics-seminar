\section{Anonymous Online Trading}
\label{trading}

Trading goods involves at least two entities. Think of them as vendor and consumer. They could again consist of more than one entity, but for simplicity reasons, this paper does not consider a possible multitude. More entities are involved, if the traders use trading platforms or payment providers. In the case of online trading, not only the direct participants have to be considered, but also for example the internet service providers (ISPs) or trading platform operators. Governments are also involved to some extent, as both -- vendor and consumer -- could live in countries where some regulations on trading are set up.
This section will elaborate on the question, what it means for a consumer or a vendor to be anonymous. The consequences for other holders of interest are discussed. The focus is set on deep web markets, especially Silk Road.

\subsection{Anonymous Entities}

Each entity involved in a trading process could be anonymous. In the context of black markets on the internet, such as Silk Road \cite{silkroad2013}, there are mainly three parties involved during trade: the vendor, the consumer and the platform operator. Any of which has different interests in keeping up the anonymity. Anonymity means, that the entity on the internet cannot be hold accountable for their online activities in the real world\cite{accountability2014}.

In the first place, being anonymous means being private. From a consumers perspective, anonymity is good for acting in secrecy. People who are embarrassed about some products, for example erotics, may seek anonymous market places to buy them.

As soon as illegal goods or services are involved, anonymity gets a different level of importance. Staying anonymous brings some level of safety. Safety from prosecution or law enforcement on the one hand, and safety from other competing sellers on the other hand. Thus I argue, that the success of deep web markets is mainly influenced by the ability to act anonymously.


\subsection{Trusting the Anonymity}

One of the biggest challenges for anonymous online trading is to keep up the anonymity throughout the whole trading process. A lot weak spots in the communication infrastructure have to be considered; the anonymous online communication itself, transferring money between entities and, most often, exchanging material commodities over some geographical distance. Many steps in the process can leak a traders real identity.

For a business success, the involved entities have to trust either each other or in the trading process itself. Trust is a very important driver for internet trading success\cite{internetTrust2004}.

Deep web platforms like Silk Road\cite{silkroad2013} combine many technologies to enable anonymity for all involved parties. Trust in generally available technologies is something that the platform operators do not have to build up themselves. By using trustworthy technologies, one can accomplish at least some level of trustworthiness, in terms of communication security. Many deep web stores use the Tor network\footnote{\url{https://www.torproject.org}} to facilitate anonymous connections for all entities\cite{silkroad2013}. The most commonly used payment method is some cryptocurrency exchange, for example Bitcoin. On Silk Road, even the prices for the products were directly displayed in Bitcoin. Furthermore, Silk Road provided recommendations for securing the internet connection and how to exchange wares anonymously. They recommend to use ``delivery addresses that are distinct to a buyer's residence'' \cite{silkroad2013}. But ultimately it is the traders choice to trust a platform and also their own responsibility to install proper mechanisms to keep up anonymity.

\subsection{Effects of Anonymity during Trade}

Staying anonymous during trade has many effects, apart from the directly visible effects like safety and privacy. Karen Frost-Arnold \cite{accountability2014} states the theory, that anonymity on the internet can be seen as a driver for diversity -- in a context of free speech, blogging or online encyclopedias. Accountability on the other hand diminishes diversity \cite{accountability2014}. To some extent, I feel that is applicable to anonymous online markets.

Silk Road offered a diverse variety of goods. In the context of deep web markets, this diversity of goods often comes along with their illegality or immorality\cite{silkroad2013}. We can compare that to legal, non-anonymous online markets such as Amazon. Few to none of the sold commodities on Silk Road are sold on normal web shops. I argue that is because of the anonymous trading possibilities that are only available on certain deep web shops. I conclude a similar result as in \cite{accountability2014}, anonymity is indeed a driver for -- admittedly questionable -- diversity. 

This has different effects for many participants. Governments typically seek to control illegal goods and their distribution, in especially weapons or drugs. Depending on how well all traders secured their anonymity, prosecuting such trades becomes very difficult for governments. Except from illegality of goods, the trade itself cannot be controlled. Consumers have no law enforcement backing them, if they get ripped off by some seller.

From a vendors perspective, anonymous online trading allows for selling questionable goods, that would have been difficult to sell otherwise. They can address a broader range of customers. Furthermore, they can do so quite openly without the fear of prosecution.

From a consumers perspective, anonymous online trading offers the possibility to retrieve commodities, that are not accessible easily in the offline world. On the other hand, the anonymity of sellers directly influences the trust that consumers have towards the sellers. As trust is very important for internet trading success\cite{internetTrust2004}, other mechanisms  have to be used to reestablish consumer trust on anonymous markets. Section~\ref{consumers} discusses trust among anonymous traders in detail.