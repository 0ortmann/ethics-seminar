\section{Consumer Ethics}

Most online trading does not happen exactly simulatneously, but rather one party -- the consumer -- pays in advance. If both entities are truly anonymous, there is no guarantee that the consumer will really receive the goods she paid for. That is because no thrid party could enforce justice; vendors could just take the money without sending anything. Nonetheless, the abscence of regulations for consumer safety does not hinder anonymous trading. And surprisingly enough, anonymous markets work -- in a sense that consumers do receive the goods they paid for\cite{silkroad2013}. This section collects some ethical thoughts about consumers, trust and justice. 



% There exist different ways to suggest trustworthyness to consumers.
% Typically, anonymous online trading is initiated by consumers, who are in need for some good. They have to be able to somehow find vendors that offer the questioned good. 
% 
% - Driver to consume: goods. ?
% - Trust = somehow fundamental. Trust either in: platform, customer reviews, product, vendor, technology...
% - Consumers expect some behavior from vendor or platform. -> bolton2004
% - also, need something to rely on. trust does not just come from nothing.
% - silkroad: many users. why. customer feedbacks?
    % -- maybe apply some theories (eg. they have a feedback system, reputation, quality, service etc)
    % -- what trust is visible here, why do people dare to use silkroad.
% 
% 
% 
% 
% consuming anon trading -- what can I expect the vendor to behave like?
% \begin{itemize}
    % \item no legal liability -> no justice. paid but got no goods? happens..
    % \item silkroad/deep web -> is just without justice enforcing 3rd party
    % \item vendor reputation and trust in other peoples reports
    % \item assuming ethical attitudes -> the business I am dealing with has to somehow deal with this. assume it is interested in shareholders? stakeholders? stakeholders it feels like! why? 
% \end{itemize}