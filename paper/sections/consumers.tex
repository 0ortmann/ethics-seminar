\section{Consumer Ethics}
\label{consumers}

Most online trading does not happen exactly simultaneously, but rather one party -- the consumer -- pays in advance. If both entities are truly anonymous, there is no guarantee that the consumer will really receive the goods she paid for. This is called the ``trust dilemma''\cite{internetTrust2004}. The dilemma arises, because no third party could enforce justice; vendors could just take the money without sending anything. Nonetheless, the absence of regulations for consumer safety does not hinder anonymous trading. And surprisingly enough, anonymous markets work -- in a sense that consumers dare to buy goods and in turn really receive the goods they paid for\cite{silkroad2013}. This section collects some ethical thoughts about consumers, trust and justice. 

\subsection{Anonymous Accountability}

Consumers need to build up trust towards sellers to overcome the trust dilemma\cite{internetTrust2004}. From a consumers perspective, trust could arise if there is a visible motive for a seller to actually ship the goods. But also a trustworthy history of a sellers activities could build up trust. The challenge is to keep up anonymity for all entities.

Many regular online shops, as well as deep web shops like silkroad, employ reputation systems for sellers\cite{internetTrust2004, silkroad2013}. Consumers who bought articles can give feedback on the sellers activities. That feedback is visible for everybody else. Positive feedback could encourage trust, whilst negative feedback diminishes trust and could ultimately result in other consumers backing off from a misbehaving seller\cite{internetTrust2004}. In the example of silkroad, reputation was coupled to a sellers pseudonym\cite{silkroad2013}. Each potential consumer could review a sellers reputation and build up their own trust towards an anonymous seller. Thus, anonymity was kept up without diminishing trust\cite{internetTrust2004}.

\subsection{Trusting a Market Platform}

Trusting in other consumers feedback in turn requires trust into the reputation system itself. The reputation system is built and offered by the platform, like Amazon or silkroad. I find it very interesting, why consumers decide to trust in a platform. I could not find clear studies about consumers choice for a particular platform. So I argue, that trust into a platform is built up via different channels on the internet. I strongly believe that recommendations on some forums or discussions in blogs spread the word for ``well working'' deep markets. For example, people discuss about a platforms reliability on other platforms, like reddit\footnote{\url{https://www.reddit.com}}. I think those discussions build up trust. When consumers ultimately decide for a platform, that is often based on trust in other consumers experience, discussed on some other media than the questioned platform. I think discussions on the internet can be compared to a reputation system to some extent.