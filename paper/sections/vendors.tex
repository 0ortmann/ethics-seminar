\section{Viewpoints on Vendor Morality}

As an anonymous vendor, there is no need to actually send any goods, once a payment is received. The vendors in anonymous online markets are typically not private persons, selling one thing, but can rather be seen as a business\cite{silkroad2013}. When running a business, one may consider different holders of interest in the businesses fortune. Not sending goods after receiving payments will likely diminish consumer trust into that business. That is, if consumers do know about the misbehavior of the vendor.

The remainder of this section assumes, that there is some driver for anonymous vendors to actually send out their goods. Three well known ethical theories are reflected in the context of anonymous vendors behavior. First, Milton Friedmans shareholder theory is applied to anonymous online market vendors. Second, the stakeholder theory by Edward Freeman is discussed in context. Finally, vendor morality is considered from a utilitarian viewpoint.



%wenn jemand was verkauft, hat er stake + shareholder. da kann man viel labern, ob verkaufen dann moralisch ok ist. aus stake + shareholder sicht ja schon :D ist ja für business ethics voll egal ob dabei jemand stirbt.
%offering anon trading: (using a platform, not running it) -- should I do it as a business (-person), ethical concerns
%\begin{itemize}
%    \item shareholder theory
%    \item stakeholder theory
%    \item utilitarianism
%\end{itemize}

