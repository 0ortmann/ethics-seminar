\section{Viewpoints on Vendor Morality}
\label{vendors}

As an anonymous vendor, there is no need to actually send any goods, once a payment is received. The vendors in anonymous online markets are typically not private persons, selling one thing, but can rather be seen as a business\cite{silkroad2013}. When running a business, one may consider different holders of interest in the businesses fortune. Not sending goods after receiving payments will likely diminish consumer trust into that business. That is, if consumers do know about the misbehavior of the vendor, eg. via a reputation system.

The remainder of this section assumes, that there is some driver for anonymous sellers to actually send out their goods. It covers the question, what ethical viewpoints or motives a sellers could have, to partake on anonymous online trading. Two well known ethical theories are reflected in the context of anonymous vendors behavior. First, Milton Friedman's shareholder theory is applied to anonymous online market vendors. Second, the stakeholder theory by Edward Freeman is discussed in context.
% Finally, vendor morality is considered from a utilitarian viewpoint.

\subsection{About the Law}

Both, shareholder and stakeholder theory come with the obligation to obey the law when running a business\cite{shareholder, stakeholder}. One can argue about anonymous online trading and the law. It strongly depends on what goods are sold and to what countries they are shipped. Exactly that is the point -- there are countries, where certain drugs or the possession of weapons are legal. As the seller is anonymous, we do not know what legal obligations apply. And as the trading happens fully anonymous, a seller cannot know which legal obligations frame his customers. In a fully anonymous context, nobody can possibly know what legal constraints apply to whom.

If it is unknown what legal obligations are forced upon some entity, does that entity have to obey all possible obligations (on the world) or none? I argue that we -- as external observers -- are in no position to judge. One cannot say whether it is legally wrong to buy or sell something, without knowing the law.

This applies for all the anonymized online communication, but what is with shipping goods? Again, it is notably that we do not know with certainty, if the seller breaks a law \emph{just by sending} goods to some country. So we cannot judge in a legal sense about the seller. A consumer most probable would break a law by receiving prohibited goods.

Keeping this in mind, we will discuss if anonymous sellers can act ethically.

\subsection{Shareholder Theory}

Mitlon Friedman proposed the shareholder theory in 1970. It basically states that the main responsibility of a business is to maximize its own profit. That is, as long the business obeys the law and ``ethical custom'' of the society\cite{shareholder}. 

Maximizing profit in the context of anonymous online trading, maybe on a deep web platform like silkroad, is directly dependent on building up positive reputation. As pointed out in section~\ref{consumers}, positive reputation contributes to consumer trust, which is essential to keep an anonymous business running. That means, it is a sellers intrinsic motivation to make the consumers happy in order to keep up the business. Especially by shipping the goods.

Thus, from a shareholders perspective, an anonymous online seller acts ethically, as long as the goods are shipped and the consumers are satisfied. I think this is one of the main reasons why anonymous online trading works in general.

\subsection{Stakeholder Theory}

In 1984 Edward Freeman proposed the stakeholder theory\cite{stakeholder}. It includes not just shareholders, but any entity affected by business actions, such as employees or customers. Countries, or more precisely governments can be considered stakeholders on any kind of trade. The import of banned substances or weapons violates many governments interests. Generally, any import may concern a government, because tax fees may apply. Thus we can agree, that governments are stakeholders on anonymous online sellers. Customers are stakeholders as well. If a seller obeys governments interests and does not send any goods, that would conflict with the customers interest.

For anonymous online trading, most of the time sellers do not know in advance which country a consumer belongs to. For example on silkroad the shipping information becomes available after payment. A customer sends the money and some encrypted message where to ship to\cite{silkroad2013}. Once the money is received, the seller is left with the ethical conflict to eventually send goods to a country where they are banned. The only way to avoid this potential conflict is to avoid selling goods to anonymous customers. Or to only sell goods that are definitely neither prohibited nor taxed by any country in the world, where potential customers could live.

Considering all potential stakeholders interests, it is inherently impossible for a seller to offer goods -- that are at least banned in one country on the world -- to an anonymous world wide audience. Anonymization during the trading process makes it impossible for a seller to exclude certain customers before the trade. Consequently, anonymous online trading in the deep web can be considered unethical from a stakeholders perspective.