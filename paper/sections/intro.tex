\section{Introduction}

% - Trading in the internet, bitcoins, deepweb
% - Overview: what comes in the sections of this paper

% TODO ref sections

The modern internet enables people -- or entities -- to do many things without the need of physical contact to others or even the real world. Buying and selling goods on the internet is well adopted and appreciated by throusands. Huge platforms like amazon or ebay and many more enable entities to partake of online trading. It is possible to transfer money online and it can be changed to various currencies. Services like paypal and technologies like cryptocurrencies facilitate cashflow without any physical interaction.

A more general property of the internet is that it enables anonymity. Of course, one is not anonymous everywhere and most of the time true anonymity is not the default. In the context of this paper, anonymity means that some entity on the internet cannot be connected to their real world or ``offline'' identity. There exist many technologies that focus on providing anonymity. Be it anonymous connections or messages between entities via the Tor network or untraceable cashflow between entities.

Many ethical questions arise when anonymity and trading are combined on the internet. The remainder of this work is as follows. Section 1 describes general aspects of trading and anonymity on the internet. Opportunities and pitfalls are outlined. Section 2 and 3 deal with moral implications for either consumers or vendors. Different ethical viewpoints are taken and evaluated. Section 4 examines ethical concerns, when some entity enables others to trade anonymously. Platform hoster who explicitly encourage anonymous trading and businesses who develop technologies that support it are discussed critically.